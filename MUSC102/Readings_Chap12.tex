% Options for packages loaded elsewhere
\PassOptionsToPackage{unicode}{hyperref}
\PassOptionsToPackage{hyphens}{url}
%
\documentclass[
  12pt,
]{article}
\usepackage{amsmath,amssymb}
\usepackage{iftex}
\ifPDFTeX
  \usepackage[T1]{fontenc}
  \usepackage[utf8]{inputenc}
  \usepackage{textcomp} % provide euro and other symbols
\else % if luatex or xetex
  \usepackage{unicode-math} % this also loads fontspec
  \defaultfontfeatures{Scale=MatchLowercase}
  \defaultfontfeatures[\rmfamily]{Ligatures=TeX,Scale=1}
\fi
\usepackage{lmodern}
\ifPDFTeX\else
  % xetex/luatex font selection
\fi
% Use upquote if available, for straight quotes in verbatim environments
\IfFileExists{upquote.sty}{\usepackage{upquote}}{}
\IfFileExists{microtype.sty}{% use microtype if available
  \usepackage[]{microtype}
  \UseMicrotypeSet[protrusion]{basicmath} % disable protrusion for tt fonts
}{}
\makeatletter
\@ifundefined{KOMAClassName}{% if non-KOMA class
  \IfFileExists{parskip.sty}{%
    \usepackage{parskip}
  }{% else
    \setlength{\parindent}{0pt}
    \setlength{\parskip}{6pt plus 2pt minus 1pt}}
}{% if KOMA class
  \KOMAoptions{parskip=half}}
\makeatother
\usepackage{xcolor}
\setlength{\emergencystretch}{3em} % prevent overfull lines
\providecommand{\tightlist}{%
  \setlength{\itemsep}{0pt}\setlength{\parskip}{0pt}}
\setcounter{secnumdepth}{-\maxdimen} % remove section numbering
\newcommmand{\boldx}{\mathbf{x}}
\ifLuaTeX
  \usepackage{selnolig}  % disable illegal ligatures
\fi
\IfFileExists{bookmark.sty}{\usepackage{bookmark}}{\usepackage{hyperref}}
\IfFileExists{xurl.sty}{\usepackage{xurl}}{} % add URL line breaks if available
\urlstyle{same}
\hypersetup{
  pdftitle={Options for packages loaded elsewhere},
  hidelinks,
  pdfcreator={LaTeX via pandoc}}

\title{Options for packages loaded elsewhere}
\author{}
\date{}

\begin{document}
\maketitle

\PassOptionsToPackage{unicode}{hyperref}
\PassOptionsToPackage{hyphens}{url}

\%

\documentclass[
  12pt,
]{article}
\usepackage{amsmath,amssymb}
\usepackage{iftex}
\ifPDFTeX
  \usepackage[T1]{fontenc}
  \usepackage[utf8]{inputenc}
  \usepackage{textcomp} % provide euro and other symbols
\else % if luatex or xetex
  \usepackage{unicode-math} % this also loads fontspec
  \defaultfontfeatures{Scale=MatchLowercase}
  \defaultfontfeatures[\rmfamily]{Ligatures=TeX,Scale=1}
\fi
\usepackage{lmodern}
\ifPDFTeX\else
  % xetex/luatex font selection
\fi
% Use upquote if available, for straight quotes in verbatim environments
\IfFileExists{upquote.sty}{\usepackage{upquote}}{}
\IfFileExists{microtype.sty}{% use microtype if available
  \usepackage[]{microtype}
  \UseMicrotypeSet[protrusion]{basicmath} % disable protrusion for tt fonts
}{}
\makeatletter
\@ifundefined{KOMAClassName}{% if non-KOMA class
  \IfFileExists{parskip.sty}{%
    \usepackage{parskip}
  }{% else
    \setlength{\parindent}{0pt}
    \setlength{\parskip}{6pt plus 2pt minus 1pt}}
}{% if KOMA class
  \KOMAoptions{parskip=half}}
\makeatother
\usepackage{xcolor}
\setlength{\emergencystretch}{3em} % prevent overfull lines
\providecommand{\tightlist}{%
  \setlength{\itemsep}{0pt}\setlength{\parskip}{0pt}}
\setcounter{secnumdepth}{-\maxdimen} % remove section numbering
\ifLuaTeX
  \usepackage{selnolig}  % disable illegal ligatures
\fi
\IfFileExists{bookmark.sty}{\usepackage{bookmark}}{\usepackage{hyperref}}
\IfFileExists{xurl.sty}{\usepackage{xurl}}{} % add URL line breaks if available
\urlstyle{same}
\hypersetup{
  hidelinks,
  pdfcreator={LaTeX via pandoc}}

\author{}
\date{}

\begin{document}

\% Readings for Chapter 1 and 2 \% MUSC 102

\chapter{Reading}\label{reading}

\section{Chapter 1}\label{chapter-1}

Qur'an Recitation is not considered music. Our ears might make us think ``this is music''. But we need to ask what we think of it as.

Thus: \textbf{Music is \emph{not} a universal language.}

\subsection{\texorpdfstring{\emph{5 propositions} that attempt to understand what music is}{5 propositions that attempt to understand what music is}}\label{propositions-that-attempt-to-understand-what-music-is}

\subsubsection{1. All music is sound}\label{all-music-is-sound}

\begin{description}
%
  \setlength{\itemsep}{0pt}\setlength{\parskip}{0pt}
\item[tone]
a musical sound. If a sound is musical, it is a tone
\end{description}

Anybody can define a tone as however they want, so some can say something is music while other say it's not

\subsubsection{2. The sounds that comprise musical works is organized in some way}\label{the-sounds-that-comprise-musical-works-is-organized-in-some-way}

\subsubsection{3. Music is a humanly organized sound}\label{music-is-a-humanly-organized-sound}

Basically, animals do not sing. They ``sing'' because we attribute that word because of we're humans.

There is research that animals \emph{DO} conceptualize certain sounds as how people conceptualize humans. We're just not saying that it is for this class.

\subsubsection{4. Music is a product of human intention and perception}\label{music-is-a-product-of-human-intention-and-perception}

\begin{description}
\item[HIP (human intention and perception) approach]
\begin{enumerate}
\def\labelenumi{(\arabic{enumi})}
%
  \setlength{\itemsep}{0pt}\setlength{\parskip}{0pt}
\item[]
\item
  privileges inclusiveness over exclusiveness (2) music is inseparable from the people who make it or experience it
\end{enumerate}
\end{description}

According to this model, \emph{4'33'\,'} is a piece of music because the composer made with the intention of music, the performers perform with the intention of music, and some of the audience considers it music. According to HIP, it \emph{is} music

\subsubsection{5. Music, the term is tied to Western culture and assumptions}\label{music-the-term-is-tied-to-western-culture-and-assumptions}

\begin{description}
%
  \setlength{\itemsep}{0pt}\setlength{\parskip}{0pt}
\item[ethnocentrism]
imposing of our own culturally grounded beliefs, biases, practices (\emph{When you believe your nation is at the center of everything})
\end{description}

Some countries don't even have the concept of music.

\begin{center}\rule{0.5\linewidth}{0.5pt}\end{center}

\section{Chapter 2}\label{chapter-2}

\begin{description}
%
  \setlength{\itemsep}{0pt}\setlength{\parskip}{0pt}
\item[ethnomusicology]
the interdisciplinary academic field that draws on musicology, anthropology, other disciplines to study world music
\item[the musicultural phenomenon]
the phenomenon where \emph{music as sound} and \emph{music as culture} are mutually reinforcing
\end{description}

\subsection{Culture in Music}\label{culture-in-music}

\begin{description}
%
  \setlength{\itemsep}{0pt}\setlength{\parskip}{0pt}
\item[culture]
the complex whole which includes knowledge, art, belief, law, morals, custom, any other capailities and habits acquired by man as members of society \emph{(communities)}
\end{description}

\texttt{Music\ is\ a\ mode\ of\ cultural\ production\ and\ representation.}

\subsection{Meaning in Music}\label{meaning-in-music}

Music isn't music until a meaning is connected to the sound

The meaning can be found in two ways:

\begin{enumerate}
\def\labelenumi{\arabic{enumi}.}
%
  \setlength{\itemsep}{0pt}\setlength{\parskip}{0pt}
\item
  Meaning relative to one another
\end{enumerate}

\begin{itemize}
\item
  \begin{description}
  \tightlist
  \item[notes]
  specific tones in a piece of music
  \end{description}
\item
  \begin{description}
  \tightlist
  \item[pitch]
  relative highness and lowness of notes
  \end{description}
\end{itemize}

\begin{enumerate}
\def\labelenumi{\arabic{enumi}.}
\setcounter{enumi}{1}
%
  \setlength{\itemsep}{0pt}\setlength{\parskip}{0pt}
\item
  Meaning that transcends musical piece itself
\end{enumerate}

\emph{ex)} What if Mary had a Little Lamb was a sad funeral song?

\subsection{Identity in Music}\label{identity-in-music}

\begin{description}
%
  \setlength{\itemsep}{0pt}\setlength{\parskip}{0pt}
\item[identity]
idea of who they are and what units/distinguishes them ffrom other peoples/entities
\end{description}

Music provides answers to \emph{who am I?} and \emph{who are we?}. It also provides answers to \emph{who is she/he?} and \emph{who are they?}.

\emph{Rabbit Dance} (song)

\begin{description}
%
  \setlength{\itemsep}{0pt}\setlength{\parskip}{0pt}
\item[vocables]
a term used to describe nonlinguistic syllables (\emph{ex} scatting)
\end{description}

\subsubsection{Societies}\label{societies}

\begin{description}
%
  \setlength{\itemsep}{0pt}\setlength{\parskip}{0pt}
\item[society]
group of persons regarded as forming as single community of related interdependent individuals
\item[social institutions]
a group that gathers for some social purpose (\emph{ex} curches, sororieties/fraternities, synagogues)
\end{description}

Focus is the impact of musicians/musical institutions on societies. (\emph{ex} Bali's gemelan club. First male only, now women can play too).

\subsubsection{Cultures}\label{cultures}

society is defined by its social institutions, \textbf{culture} is defined by its collective worldview shared by its members. Cultures are rooted in beliefs, ideas, and practices.

\emph{Example}: belenganjur performers are done by male because it is believed it's to ward off evil spirits and males have the strength to do so. The government allowed for women to participate in belenganjur groups of their own but I Wayan Beratha views this as reprehensible.

\subsubsection{Nations and nation-states}\label{nations-and-nation-states}

\begin{description}
%
  \setlength{\itemsep}{0pt}\setlength{\parskip}{0pt}
\item[nation-state]
it shares a national society, culture and a \emph{homeland}
\end{description}

\emph{ex} Canada

\begin{description}
%
  \setlength{\itemsep}{0pt}\setlength{\parskip}{0pt}
\item[nation]
they share a society and culture, but \emph{no homeland}
\end{description}

\emph{ex} Palestine

\begin{description}
%
  \setlength{\itemsep}{0pt}\setlength{\parskip}{0pt}
\item[Nationalist Music]
often promoted by government (or some official institution) to symbolize a ``national identity''
\end{description}

Nationalist music share a feature of nation-building or agenda.

\texttt{As\ surely\ as\ music\ has\ the\ power\ to\ reinforce\ national\ solidarity\ and\ ideals,\ it\ also\ has\ the\ power\ to\ profoundly\ challenge\ and\ undermine\ them}

\subsubsection{Diasporas and transnational communities}\label{diasporas-and-transnational-communities}

\begin{description}
%
  \setlength{\itemsep}{0pt}\setlength{\parskip}{0pt}
\item[diaspora]
refers to the international network of communities linked together by identification with a common ancestral homeland and culture
\end{description}

(People in diaspora live away from their ``homeland'' without guarantee that they will return)

\emph{ex} Jewish, African, Irish diaspora

And then there's **virtual communities* (people online)

\subsection{The Individual in music}\label{the-individual-in-music}

\begin{description}
%
  \setlength{\itemsep}{0pt}\setlength{\parskip}{0pt}
\item[musical syncretism]
merging of formerly distinct styles and and idioms into new forms of expression
\item[fieldwork]
living for an extended period of time among people whose lives and music one researches, often learning and performing the music themselves
\end{description}

\subsection{Spirituality and Transcendence in Music}\label{spirituality-and-transcendence-in-music}

\emph{ex}

\begin{itemize}
\item
  During Balinese cremations, the souls rise up the ladders of music
\item
  Baal Shem Tov rose to heaven through music and became music
\item
  Hindu music are cyclical and reflect their Hindu beliefs about the design of the universe
\end{itemize}

\subsection{Music and Dance}\label{music-and-dance}

Dance can show off culture and cultural identity, so it can also show some if the culture's more troubling issues of gender and race.

\emph{Ex}

\begin{itemize}
\item
  In the middle east, professional female dancers are considered low class
\item
  Africans were considered ``primitive'' and this was rationalized by their music and dance
\end{itemize}

\begin{description}
%
  \setlength{\itemsep}{0pt}\setlength{\parskip}{0pt}
\item[rituals]
special events during which individuals or communities enact their core beliefs and values
\end{description}

\subsection{Music as a Commodity and the Patronage of Music}\label{music-as-a-commodity-and-the-patronage-of-music}

\textbf{Ownership of Music}

\emph{ex} Similar to the West's \emph{copyright} stuff. Aboriginal Australian/Amerindians also are only allowed to perform music of their own, and not others.

\begin{description}
%
  \setlength{\itemsep}{0pt}\setlength{\parskip}{0pt}
\item[Music patronage]
involves support in music and music institutions
\end{description}

\subsection{Transmission of Music and Musical Knowledge}\label{transmission-of-music-and-musical-knowledge}

\textbf{a.k.a. The sharing of music}

\subsubsection{Production and Reception}\label{production-and-reception}

All music transmission has the production and reception. In Africa, this is not so clear. All members are expected to participate and encourage performances.

Music can be taught structurally, or via a learning \emph{osmosis}

\subsubsection{Music creation process}\label{music-creation-process}

\begin{enumerate}
\def\labelenumi{\arabic{enumi}.}
%
  \setlength{\itemsep}{0pt}\setlength{\parskip}{0pt}
\item
  Composition
\end{enumerate}

planning before the performance

\begin{enumerate}
\def\labelenumi{\arabic{enumi}.}
\setcounter{enumi}{1}
%
  \setlength{\itemsep}{0pt}\setlength{\parskip}{0pt}
\item
  Interpretation
\end{enumerate}

Music listeners make sense of the composition in their own performance

\begin{enumerate}
\def\labelenumi{\arabic{enumi}.}
\setcounter{enumi}{2}
%
  \setlength{\itemsep}{0pt}\setlength{\parskip}{0pt}
\item
  Improvisation
\end{enumerate}

Composing in the middle of the performance

\begin{enumerate}
\def\labelenumi{\arabic{enumi}.}
\setcounter{enumi}{3}
%
  \setlength{\itemsep}{0pt}\setlength{\parskip}{0pt}
\item
  Arranging
\end{enumerate}

Taking existing music and making it your own

\subsection{Music in the Process of Tradition}\label{music-in-the-process-of-tradition}

\begin{description}
%
  \setlength{\itemsep}{0pt}\setlength{\parskip}{0pt}
\item[tradition (in this book)]
conceived as a \textbf{\emph{process}} of creative transformation. Its remarkable feature is the continuity it nurtures and sustains
\end{description}

\texttt{what\ qualifies\ music\ as\ traditional\ is\ not\ how\ old\ it\ is,\ but\ rather\ how\ well\ it\ teaches,\ reinforces,\ and\ creates\ the\ social\ values\ of\ its\ producers\ and\ consumers.\ Traditional\ music\ is\ not\ something\ that\ is\ stuck\ in\ the\ past;\ it\ grows\ and\ changes,\ just\ as\ the\ people\ who\ make\ and\ listen\ to\ it\ grow\ and\ change,\ just\ as\ the\ values\ they\ share\ with\ those\ close\ to\ them\ changes.\ -\ Henry\ Spiller}

``Traditional Music'' -\textgreater{} ``Music of Tradition''

Music of tradition can be modern, radical, experimental (as much as it can be archaic and ancient)

\end{document}

\end{document}
