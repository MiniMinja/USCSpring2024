% Options for packages loaded elsewhere
\PassOptionsToPackage{unicode}{hyperref}
\PassOptionsToPackage{hyphens}{url}
%
\documentclass[
  12pt,
]{article}
\usepackage{amsmath,amssymb}
\usepackage{iftex}
\ifPDFTeX
  \usepackage[T1]{fontenc}
  \usepackage[utf8]{inputenc}
  \usepackage{textcomp} % provide euro and other symbols
\else % if luatex or xetex
  \usepackage{unicode-math} % this also loads fontspec
  \defaultfontfeatures{Scale=MatchLowercase}
  \defaultfontfeatures[\rmfamily]{Ligatures=TeX,Scale=1}
\fi
\usepackage{lmodern}
\ifPDFTeX\else
  % xetex/luatex font selection
\fi
% Use upquote if available, for straight quotes in verbatim environments
\IfFileExists{upquote.sty}{\usepackage{upquote}}{}
\IfFileExists{microtype.sty}{% use microtype if available
  \usepackage[]{microtype}
  \UseMicrotypeSet[protrusion]{basicmath} % disable protrusion for tt fonts
}{}
\makeatletter
\@ifundefined{KOMAClassName}{% if non-KOMA class
  \IfFileExists{parskip.sty}{%
    \usepackage{parskip}
  }{% else
    \setlength{\parindent}{0pt}
    \setlength{\parskip}{6pt plus 2pt minus 1pt}}
}{% if KOMA class
  \KOMAoptions{parskip=half}}
\makeatother
\usepackage{xcolor}
\setlength{\emergencystretch}{3em} % prevent overfull lines
\providecommand{\tightlist}{%
  \setlength{\itemsep}{0pt}\setlength{\parskip}{0pt}}
\setcounter{secnumdepth}{-\maxdimen} % remove section numbering
"\\newcommmand{\\boldx}{\\mathbf{x}}"
\ifLuaTeX
  \usepackage{selnolig}  % disable illegal ligatures
\fi
\IfFileExists{bookmark.sty}{\usepackage{bookmark}}{\usepackage{hyperref}}
\IfFileExists{xurl.sty}{\usepackage{xurl}}{} % add URL line breaks if available
\urlstyle{same}
\hypersetup{
  pdftitle={Readings for Chapter 3 - 4},
  pdfauthor={MUSC 102},
  hidelinks,
  pdfcreator={LaTeX via pandoc}}

\title{Readings for Chapter 3 - 4}
\author{MUSC 102}
\date{}

\begin{document}
\maketitle

\chapter{Chapter 3}\label{chapter-3}

\section{The 4 Basic Properties of Tones}\label{the-4-basic-properties-of-tones}

\begin{description}
\tightlist
\item[Duration]
How long a tones is
\item[Frequency]
How high or low a tone is in pitch
\item[Amplitude]
How loud or soft a tone is
\item[Timbre]
sound quality/color
\end{description}

\section{Rhythm}\label{rhythm}

\begin{description}
\tightlist
\item[rhythm]
how the sounds and silences are organized in time
\end{description}

\begin{itemize}
\tightlist
\item
  \textbf{Sixteenth notes, eight notes, quarter notes}
\end{itemize}

\subsection{Beat}\label{beat}

: underlying pulse

\subsection{Subdivision}\label{subdivision}

: when beats are divided into smaller rhythmic units

\subsection{Meter}\label{meter}

: number of beats in a \textbf{measure}

\begin{description}
\tightlist
\item[measure]
grouping of beats in Western music
\end{description}

\emph{ex}) Alphabet Music is a meter of 4

\subsection{Accent and Syncopation}\label{accent-and-syncopation}

\begin{description}
\tightlist
\item[accent]
notes that are given a little more \emph{oomph}, emphasis
\item[syncopation]
accented notes that fall in-between the beats
\end{description}

\subsection{Tempo}\label{tempo}

\begin{itemize}
\tightlist
\item
  The rate at which beats pass
\end{itemize}

\subsection{Free Rhythm}\label{free-rhythm}

\begin{itemize}
\tightlist
\item
  music with no discernible beat, seeming to float across time instead of march with it
\end{itemize}

\begin{description}
\tightlist
\item[metric music]
music with discernible meter/tempo
\end{description}

\chapter{Chapter 4}\label{chapter-4}

\section{Pitch and Melody}\label{pitch-and-melody}

\begin{description}
\tightlist
\item[Melody]
the particular sequence of pitches
\end{description}

Distinct Features:

\begin{enumerate}
\def\labelenumi{\arabic{enumi}.}
\item
  \begin{description}
  \tightlist
  \item[\textbf{Melodic Range}]
  the distance from the lowest note to the highest
  \end{description}
\item
  \begin{description}
  \tightlist
  \item[\textbf{Melodic Direction}]
  the upward/downward movement of the melody
  \end{description}
\item
  \begin{description}
  \tightlist
  \item[\textbf{Melodic Contour}]
  the overall ``shape'' of the music, a product of \emph{range} and \emph{direction} and other things
  \end{description}
\end{enumerate}

\subsection{Names of Pitches}\label{names-of-pitches}

\begin{itemize}
\tightlist
\item
  A
\item
  B
\item
  C
\item
  D
\item
  E
\item
  F
\item
  G
\end{itemize}

\begin{description}
\tightlist
\item[determinant pitch]
being able to discern the exact pitch when played on an instrument
\end{description}

\emph{ex}) piano, xylophone

\begin{description}
\tightlist
\item[indeterminant pitch]
not being able to discern the exact pitch
\end{description}

\emph{ex}) drums, triangle, cymbals

\subsection{The Western pitch system and the octave}\label{the-western-pitch-system-and-the-octave}

The 12 pitches in Western music are on the piano

\begin{description}
\tightlist
\item[scale]
ascending/descending series of notes of different pitches
\item[octave]
the same pitch of a note, but higher or lower
\end{description}

\subsection{Common scales in Western Music}\label{common-scales-in-western-music}

\begin{description}
\tightlist
\item[interval]
the distance between any two notes
\item[\emph{\textbf{Major Scale}}]
The white keys starting with C
\item[\emph{\textbf{Pentatonic Scale}}]
The keys C D E G A starting with C
\item[\emph{\textbf{Minor Scale}}]
like major scale, but the third key is typically down a half-step
\item[\emph{\textbf{Blues Scale}}]
starting on C, the notes C Eb F F\# G Bb
\end{description}

\subsection{Scales in non-Western music systems}\label{scales-in-non-western-music-systems}

Gamelan music have 2 pitch systems, slendro and pelog, but they have nothing to do with Western 12 pitch system. They are pentatonic, but not the same pentatonic as Western music.

Arab traditional music is built from

\begin{description}
\tightlist
\item[microtones]
tiny intervals.
\item[articulation]
examples include
\item[legato]
sustained notes
\item[staccato]
clipped notes
\end{description}

\section{Pitch, Chords, Harmony}\label{pitch-chords-harmony}

\begin{description}
\tightlist
\item[chord]
when two or more notes are played simultaneously
\item[harmony]
when a chord ``makes sense'' in the context of the overall piece of music
\item[chord progression]
the product of moving from one chord to another
\item[harmonization]
the result of when each notes becomes the basis of its own chord
\item[arpeggio]
the ``broken chord'', each tone presented one at a time
\end{description}

\end{document}
